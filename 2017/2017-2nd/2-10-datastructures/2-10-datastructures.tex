\documentclass{tufte-handout}

\usepackage{hyperref}

\usepackage{CJKutf8}

% \geometry{showframe}% for debugging purposes -- displays the margins
\usepackage{amsmath}

% Set up the images/graphics package
\usepackage{graphicx}
\setkeys{Gin}{width=\linewidth,totalheight=\textheight,keepaspectratio}
% \graphicspath{{figs/}}

% The following package makes prettier tables.  We're all about the bling!
\usepackage{booktabs}

% The units package provides nice, non-stacked fractions and better spacing for units.
\usepackage{units}

% The fancyvrb package lets us customize the formatting of verbatim environments.  We use a slightly smaller font.
\usepackage{fancyvrb}
\fvset{fontsize=\normalsize}

% Small sections of multiple columns
\usepackage{multicol}

% Provides paragraphs of dummy text
\usepackage{lipsum}

% These commands are used to pretty-print LaTeX commands
\newcommand{\doccmd}[1]{\texttt{\textbackslash#1}}% command name -- adds backslash automatically
\newcommand{\docopt}[1]{\ensuremath{\langle}\textrm{\textit{#1}}\ensuremath{\rangle}}% optional command argument
\newcommand{\docarg}[1]{\textrm{\textit{#1}}}% (required) command argument
\newenvironment{docspec}{\begin{quote}\noindent}{\end{quote}}% command specification environment
\newcommand{\docenv}[1]{\textsf{#1}}% environment name
\newcommand{\docpkg}[1]{\texttt{#1}}% package name
\newcommand{\doccls}[1]{\texttt{#1}}% document class name
\newcommand{\docclsopt}[1]{\texttt{#1}}% document class option name

%%%%%%%%%%%%%%%%%%%%%%%%%%%%%%%%%%%
% File: hw-preamble.tex
%%%%%%%%%%%%%%%%%%%%%%%%%%%%%%%%%%%

% Set fonts commands
\newcommand{\song}{\CJKfamily{song}} 
\newcommand{\hei}{\CJKfamily{hei}} 
\newcommand{\kai}{\CJKfamily{kai}} 
\newcommand{\fs}{\CJKfamily{fs}}

% colors
\newcommand{\red}[1]{\textcolor{red}{#1}}
\newcommand{\blue}[1]{\textcolor{blue}{#1}}
\newcommand{\teal}[1]{\textcolor{teal}{#1}}

% math
\newcommand{\set}[1]{\{#1\}}

% For math
\usepackage{amsmath}

% Define theorem-like environments
\usepackage[amsmath, thmmarks]{ntheorem}

\theoremstyle{break}
\theorembodyfont{\song}
\theoremseparator{}
\newtheorem*{problem}{Problem}

\theorempreskip{2.0\topsep}
\theoremheaderfont{\kai\bfseries}
\theoremseparator{:}
\newtheorem*{solution}{Solution}

\theoremstyle{break}
\theorempostwork{\bigskip\hrule}
\theoremsymbol{\ensuremath{\Box}}
\newtheorem*{proof}{Proof}

% algorithms
\usepackage[]{algorithm}
\usepackage[noend]{algpseudocode} % noend
% See [Adjust the indentation whithin the algorithmicx-package when a line is broken](https://tex.stackexchange.com/a/68540/23098)
\newcommand{\algparbox}[1]{\parbox[t]{\dimexpr\linewidth-\algorithmicindent}{#1\strut}}
\newcommand{\hStatex}[0]{\vspace{5pt}}
\makeatletter
\newlength{\trianglerightwidth}
\settowidth{\trianglerightwidth}{$\triangleright$~}
\algnewcommand{\LineComment}[1]{\Statex \hskip\ALG@thistlm \(\triangleright\) #1}
\algnewcommand{\LineCommentCont}[1]{\Statex \hskip\ALG@thistlm%
  \parbox[t]{\dimexpr\linewidth-\ALG@thistlm}{\hangindent=\trianglerightwidth \hangafter=1 \strut$\triangleright$ #1\strut}}
\makeatother

% For figures
% for fig with caption: #1: width/size; #2: fig file; #3: fig caption
\newcommand{\fig}[3]{
  \begin{figure}[htp]
    \centering
      \includegraphics[#1]{#2}
      \caption{#3}
  \end{figure}
}

% for fig without caption: #1: width/size; #2: fig file
\newcommand{\fignocaption}[2]{
  \begin{figure}[htp]
    \centering
    \includegraphics[#1]{#2}
  \end{figure}
}

\title{2-10 The mutual realization of stack and queue}
\author[StardustDL]{杜星亮 {\normalsize (stardustdl@163.com)}}
\date{2018/8/1}

\begin{document}

\begin{CJK*}{UTF8}{gbsn}

\maketitle

\begin{abstract}
\noindent 本文介绍使用栈实现队列和使用队列实现栈的方法。为使问题更加有意义,我们将尽量选择一个较优的实现方法。
\footnote{\href{https://stardustdl.github.io/ProblemSolving/2018/08/28/The-mutual-realization-of-stack-and-queue/}{如发现错误,欢迎指正,网页版链接}}
\end{abstract}

\section{第0节:问题引入}
\begin{itemize}
    \item 栈:先进后出的线性结构,仅允许对栈顶进行添加(push),删除(pop),访问(peek)操作,空间复杂度线性,单次操作时间复杂度为常数
    \item 队列:先进先出线性结构,仅允许对队尾进行添加(enqueue)操作,以及对队首进行删除(dequeue),访问(peek)操作,空间复杂度线性,单次操作时间复杂度为常数
    \item 需要解决的问题,在使用常数个额外空间的条件下:
    \begin{itemize}
        \item 使用两个栈尽可能高效地实现一个队列
        \item 使用两个队列尽可能高效地实现一个栈
    \end{itemize}
\end{itemize}

\section{第1节:使用栈实现队列}
\subsection{思路}
考虑到栈先进后出与队列先进先出的特点,使用一个栈 $A$ 作为队列尾,数据从这里流入;使用另一个栈 $B$ 作为队列头,数据从这里流出。我们要求先流入的数据先流出,
通过将栈 $A$ 中的元素不断弹出,并压入栈 $B$,利用先进后出特性,$B$ 的出栈顺序即 $A$ 出栈顺序的逆序,而 $A$ 出栈顺序为其入栈顺序的逆序,故 $B$ 的出栈顺序为 $A$ 的入栈顺序,即达到先进先出的效果。
综上,给出实现的伪代码。
\marginnote{栈能将输入逆序这一点很重要,之后我们还将用到}

\begin{algorithm}[t]
\caption{队列三个操作的实现}
\begin{algorithmic}[1]
    \LineComment{\teal{Move elements in $A$ to $B$}}
    \Procedure{SWAP}{$A,B$} 
    \While{$A\ne \emptyset$}
    \State $value \gets$ POP($A$)
    \State PUSH($B,value$)
    \EndWhile
    \hStatex
    \EndProcedure

    \Procedure{ENQUEUE}{$A,B,value$} 
    \State PUSH($A,value$)
    \hStatex
    \EndProcedure

    \Procedure{DEQUEUE}{$A,B$} 
    \If{$B= \emptyset$}
    \State SWAP($A,B$)
    \EndIf
    \Return POP($B$)
    \hStatex
    \EndProcedure

    \Procedure{PEEK}{$A,B$} 
    \If{$B= \emptyset$}
    \State SWAP($A,B$)
    \EndIf
    \Return PEEK($B$)
    \hStatex
    \EndProcedure
\end{algorithmic}
\end{algorithm}

\subsection{复杂度分析}
\begin{itemize}
    \item 空间复杂度:栈的空间复杂度是线性的,而且这里队列中的每一个数据仅会在两个栈中的某一个中存在,故此实现的空间复杂度为线性
    \item 时间复杂度:注意到每个数据从入队到出队只会经历:进入 $A$,离开 $A$,进入 $B$,离开 $B$ 四次移动,且每次移动复杂度为常数,故均摊复杂度为常数。
\end{itemize}

\newpage
\section{第2节:使用队列实现栈}
\subsection{思路}
\marginnote{这里遇到了个难题:利用栈可以很容易地支持翻转操作,但队列无法直接对输入序列进行顺序的改变。}
队列中一个重要特点是,我们可以通过不断删除队列头,并将其放入队列尾,实现在不影响顺序的前提下对队列中每个元素的访问,访问一遍后,我们仍可以很容易恢复到最初的队列状态。
但这一操作的弊端是,我们访问某个元素,必须将其前面的所有元素出队,这一操作的时间复杂度是最坏情况下是线性的。
\marginnote{既然我们能访问所有元素了,那只要访问最后一个就是先入后出了,很简单嘛,可是队列“滚”的次数太多了(为线性)...}
为下文叙述方便,将上述操作定义为过程“循环出入队”,即将队首出队后入队,实现队列滚动。时间复杂度由以上分析,为 $O(|Q|))$。
我们使用一个类似缓冲池的技巧:设两个队列 $Q_s,Q_a$,$Q_a$ 用于存储靠近栈顶的一部分元素,$Q_s$ 用于存储其余的元素。其中 $Q_a$ 有可变的容量上限 $cap(Q_a)$,$Q_s$ 容量无限制。
\marginnote{你可以把 $Q_s$ 看成内存(主存),把 $Q_a$ 看成 CPU 中的高速缓存}
接下来,我们依次实现栈的三个操作:
\begin{itemize}
    \item 入栈操作
    \begin{itemize}
        \item 若 $Q_a$ 不满,直接入队到 $Q_a$ 。时间复杂度:$O(1)$
        \item 若 $Q_a$ 满,将 $Q_a$ 出队,并将队列头入队到 $Q_s$。然后将待入栈元素入队到 $Q_a$。时间复杂度:$O(1)$
    \end{itemize}
    \item 出栈操作
    \begin{itemize}
        \item 若 $Q_a$ 非空,对 $Q_a$ 循环出入队,使得原队尾在队头,返回队尾,并出队。时间复杂度:$O(|Q_a|)$
        \item 若 $Q_a$ 空,对 $Q_s$ 循环出入队,使得原队尾部的 $|Q_a|+1$ 个元素出队,返回队尾,其余元素进入 $Q_a$,顺序不变。时间复杂度:$O(|Q_s|)$
    \end{itemize}
    \item 访问栈顶操作
    \begin{itemize}
        \item 若 $Q_a$ 非空,对 $Q_a$ 循环出入队,使得原队尾在队头,返回队尾,并恢复最初顺序,时间复杂度:$O(|Q_a|)$
        \item 若 $Q_a$ 空,对 $Q_s$ 循环出入队,使得原队尾部的 $|Q_a|$ 个元素出队,返回队尾,所有元素进入 $Q_a$,顺序不变。时间复杂度:$O(|Q_s|)$
    \end{itemize}
\end{itemize}
\marginnote{似乎看不出有什么改进?是的,因为我们还没指定 $cap(Q_a)$}
为让这个实现变得有效,我们先来分析其时间复杂度来自哪里:
\textbf{每个元素}从入栈到出栈经历了:
\begin{enumerate}
\item 压入:$O(1)$
\item (可选)从 $Q_a$ 到 $Q_s$:$O(1)$
\item (可选)从 $Q_s$ 到 $Q_a$:$O(|Q_s|/|Q_a|)$(最坏情况)
\item 弹出 以下两种二选一:
    \begin{itemize}
        \item 从 $Q_s$ 出队:$O(1)$
        \item 从 $Q_a$ 出队:$O(|Q_a|)$
    \end{itemize}
\end{enumerate}
接下来考虑最坏情况,那么每个元素经历入队出队共计 $O(|Q_s|/|Q_a|+|Q_a|)$.
\begin{itemize}
    \item 由 $|Q_s|/|Q_a|+|Q_a| \ge 2\sqrt{|Q_s|}$ 当且仅当 $|Q_s|/|Q_a|=|Q_a|$ 即 $|Q_a|=\sqrt{|Q_s|}$.
    \item 故最低为 $O(\sqrt{|Q_s|})$
\end{itemize}
故当我们将 $Q_a$ 的容量限制在 $\sqrt{|Q_s|}$ 时,对于 $n$ 个元素的栈,有 $|Q_s|\le n$ 总复杂度估计为 $O(n\sqrt n)$.
平均每次操作复杂度$O(n\sqrt n/(2n))=O(\sqrt n)$,由此可见,性能的确有较大提高。

\marginnote{如果你像上面那样类比了计算机中的组件,那么我们的算法做的,就可以看成从内存中读取一段,放入高速缓存中,利用高命中率来减少我们对相对低速的内存的访问}

综上,给出实现的伪代码。

\begin{algorithm}[t]
\caption{栈三个操作的实现}
\begin{algorithmic}[1]

    \Procedure{PUSH}{$Q_a,Q_s,value$} 
    \If{$|Q_a|\ge \sqrt{|Q_s|}$}
    \State ENQUEUE($Q_s$,DEQUEUE($Q_a$))
    \EndIf
    \State ENQUEUE($Q_a,value$)
    \hStatex
    \EndProcedure

    \Procedure{POP}{$Q_a,Q_s$} 
    \If{$Q_a= \emptyset$}
    \State $size \gets \sqrt{|Q_s|}$, $i \gets 0$
    \State Roll $Q_s$ until the last $size+1$ elements are at the front
    \While{$i < size$}
    \State ENQUEUE($Q_a$,DEQUEUE($Q_s$))
    \State $i \gets i+1$
    \EndWhile
    \Return DEQUEUE($Q_s$)
    \Else
    \State Roll $Q_a$ until the last element is at the front
    \Return DEQUEUE($Q_a$)
    \EndIf
    \hStatex
    \EndProcedure

    \Procedure{PEEK}{$A,B$} 
    \If{$Q_a= \emptyset$}
    \State $size \gets \sqrt{|Q_s|}$, $i \gets 0$
    \State Roll $Q_s$ until the last $size$ elements are at the front
    \While{$i < size$}
    \State ENQUEUE($Q_a$,DEQUEUE($Q_s$))
    \State $i \gets i+1$
    \EndWhile
    \State $value \gets$ DEQUEUE($Q_s$)
    \State ENQUEUE($Q_a,value$)
    \Return $value$
    \Else
    \State Roll $Q_a$ until the last element is at the front
    \State $value \gets$ DEQUEUE($Q_a$)
    \State ENQUEUE($Q_a,value$)
    \Return $value$
    \EndIf
    \hStatex
    \EndProcedure
\end{algorithmic}
\end{algorithm}

\subsection{复杂度分析}
\begin{itemize}
    \item 空间复杂度:队列的空间复杂度是线性的,而且这里栈中的每一个数据仅会在两个队列中的某一个中存在,故此实现的空间复杂度为线性
    \item 时间复杂度:通过思路中的分析,单次操作均摊复杂度为 $O(\sqrt{n})$。
\end{itemize}

\section{更多讨论}
\begin{itemize}
    \item \href{https://cstheory.stackexchange.com/questions/2562/one-stack-two-queues/5655#5655}{时间复杂度下限的讨论}
    \item \href{https://cstheory.stackexchange.com/questions/2562/one-stack-two-queues/2589#2589}{队列实现栈的 CSharp 代码实现}
\end{itemize}

\bibliography{2-10-datastructures}
\bibliographystyle{plainnat}

\end{CJK*}
\end{document}