% 1-6-programming-language.tex

%%%%%%%%%%%%%%%%%%%%
\documentclass[a4paper, justified]{tufte-handout}

%%%%%%%%%%%%%%%%%%%%%%%%%%%%%%%%%%%
% File: hw-preamble.tex
%%%%%%%%%%%%%%%%%%%%%%%%%%%%%%%%%%%

% Set fonts commands
\newcommand{\song}{\CJKfamily{song}} 
\newcommand{\hei}{\CJKfamily{hei}} 
\newcommand{\kai}{\CJKfamily{kai}} 
\newcommand{\fs}{\CJKfamily{fs}}

% colors
\newcommand{\red}[1]{\textcolor{red}{#1}}
\newcommand{\blue}[1]{\textcolor{blue}{#1}}
\newcommand{\teal}[1]{\textcolor{teal}{#1}}

% math
\newcommand{\set}[1]{\{#1\}}

% For math
\usepackage{amsmath}

% Define theorem-like environments
\usepackage[amsmath, thmmarks]{ntheorem}

\theoremstyle{break}
\theorembodyfont{\song}
\theoremseparator{}
\newtheorem*{problem}{Problem}

\theorempreskip{2.0\topsep}
\theoremheaderfont{\kai\bfseries}
\theoremseparator{:}
\newtheorem*{solution}{Solution}

\theoremstyle{break}
\theorempostwork{\bigskip\hrule}
\theoremsymbol{\ensuremath{\Box}}
\newtheorem*{proof}{Proof}

% algorithms
\usepackage[]{algorithm}
\usepackage[noend]{algpseudocode} % noend
% See [Adjust the indentation whithin the algorithmicx-package when a line is broken](https://tex.stackexchange.com/a/68540/23098)
\newcommand{\algparbox}[1]{\parbox[t]{\dimexpr\linewidth-\algorithmicindent}{#1\strut}}
\newcommand{\hStatex}[0]{\vspace{5pt}}
\makeatletter
\newlength{\trianglerightwidth}
\settowidth{\trianglerightwidth}{$\triangleright$~}
\algnewcommand{\LineComment}[1]{\Statex \hskip\ALG@thistlm \(\triangleright\) #1}
\algnewcommand{\LineCommentCont}[1]{\Statex \hskip\ALG@thistlm%
  \parbox[t]{\dimexpr\linewidth-\ALG@thistlm}{\hangindent=\trianglerightwidth \hangafter=1 \strut$\triangleright$ #1\strut}}
\makeatother

% For figures
% for fig with caption: #1: width/size; #2: fig file; #3: fig caption
\newcommand{\fig}[3]{
  \begin{figure}[htp]
    \centering
      \includegraphics[#1]{#2}
      \caption{#3}
  \end{figure}
}

% for fig without caption: #1: width/size; #2: fig file
\newcommand{\fignocaption}[2]{
  \begin{figure}[htp]
    \centering
    \includegraphics[#1]{#2}
  \end{figure}
} % feel free to modify this file
%%%%%%%%%%%%%%%%%%%%
\title{第6讲: 程序设计语言的语法与语义}
\me{魏恒峰}{hfwei@nju.edu.cn}{}{}
\date{\zhtoday} % or like 2019年9月13日
%%%%%%%%%%%%%%%%%%%%
\begin{document}
\maketitle
%%%%%%%%%%%%%%%%%%%%
\noplagiarism % always keep this line
%%%%%%%%%%%%%%%%%%%%
\begin{abstract}
  \mfig{width = 0.85\textwidth}{figs/syntax-semantics}
  \begin{center}{\fcolorbox{blue}{yellow!60}{\parbox{0.70\textwidth}{\large 
    \begin{itemize}
      \item 逻辑系统有语法、语义之分; 程序设计语言亦如是
      \item 欢迎进入\href{https://en.wikipedia.org/wiki/Programming\_language\_theory}{程序设计语言理论}的广阔世界
    \end{itemize}}}}
  \end{center}
\end{abstract}
%%%%%%%%%%%%%%%%%%%%
\beginrequired

%%%%%%%%%%%%%%%
\begin{problem}[IMP]
  考虑程序设计语言 {\bf IMP} (Imperative), 它包含如下元素:
  \begin{itemize}
    \item 整数集合 $\mathbb{Z}$ (可用 $m,n$ 表示其中的元素)
    \item 真值集合 $\mathbb{T} = \set{T, F}$
    \item 变量集合 $\mathbb{V}$ (可用 $v$ 表示其中的元素)
    \item 算法表达式 {\bf Aexp}, 支持 ``$+, -, \times$'' 三则运算
    \item 布尔表达式 {\bf Bexp}, 支持 ``$=, \le$'' 比较操作与基本的逻辑操作
    \item 语句 {\bf St}, 包括赋值语句 ($:=$)、顺序语句 ($;$)、选择语句 (if-then-else)与循环语句 (while-do)
  \end{itemize}

  \noindent 请为 {\bf IMP} 设计语法,并使用 BNF 描述。
\end{problem}

\begin{solution}
\end{solution}
%%%%%%%%%%%%%%%

%%%%%%%%%%%%%%%%%%%%
\beginoptional

%%%%%%%%%%%%%%%
\begin{problem}[``It is Ridiculous!'']
  教材 DH ``Routines as Parameters'' 章节 (第 53、54 页) 中给了一段伪代码, \\
  \begin{center}
    {``\texttt{subroutine {\bf P-with-parameter-V}}''}
  \end{center}
  并介绍了如下调用可能产生的问题: \\
  \begin{center}
    {``\texttt{call {\bf P-with-parameter-P}}''}。
  \end{center}

  \noindent {\bf 请问, 你认为是否有(或者,是否应该有)程序设计语言允许我们写出这样的代码?} 
  \begin{itemize}
    \item 如果你认为没有或者不应该有,请给出你的理由。
    \item 如果有,请给出你的可执行代码 (比如使用 C/C++, Haskell, Coq 语言等),
      并描述代码运行时的效果。
  \end{itemize}
\end{problem}

\begin{solution}
\end{solution}
%%%%%%%%%%%%%%%

%%%%%%%%%%%%%%%%%%%%
\beginot

%%%%%%%%%%%%%%%
\begin{ot}[正则表达式]
  请介绍正则表达式 (Regualr Expression) 的语法、语义与用例等。

  \noindent 基本要求:
  \begin{itemize}
    \item 循序渐进
    \item 使用有趣而实用的例子
  \end{itemize}

  \noindent 参考资料:
  \begin{itemize}
    \item \href{https://en.wikipedia.org/wiki/Regular\_expression}{Regular expression @ wiki}
    \item \href{https://regex101.com/}{regex101}
  \end{itemize}
\end{ot}

% \begin{solution}
% \end{solution}
%%%%%%%%%%%%%%%
\vspace{0.50cm}
%%%%%%%%%%%%%%%
\begin{ot}[程序设计语言的语义]
  阅读并介绍经典论文 
  \href{https://github.com/hengxin/problem-solving-class-paperswelove/tree/master/1st-semester}{``CACM1969 (Hoare) An Axiomatic Basis for Computer Progamming'' @ problem-solving-class-paperswelove}:

  \mfig{width = 0.75\textwidth}{figs/hoare}
  \begin{itemize}
    \item 作者简介 \href{https://en.wikipedia.org/wiki/Tony\_Hoare}{Tony Hoare @ wiki}
    \item 概览文章的结构与贡献
    \item 重点介绍第三节 ``Program Execution'' 的内容
    \item 介绍 Table III 中的证明示例
  \end{itemize}
\end{ot}

% \begin{solution}
% \end{solution}
%%%%%%%%%%%%%%%

%%%%%%%%%%%%%%%%%%%%
% 如果没有需要订正的题目,可以把这部分删掉

\begincorrection

%%%%%%%%%%%%%%%%%%%%

%%%%%%%%%%%%%%%%%%%%
% 如果没有反馈,可以把这部分删掉
\beginfb

% 你可以写
% ~\footnote{优先推荐 \href{problemoverflow.top}{ProblemOverflow}}:
% \begin{itemize}
%   \item 对课程及教师的建议与意见
%   \item 教材中不理解的内容
%   \item 希望深入了解的内容
%   \item $\cdots$
% \end{itemize}
%%%%%%%%%%%%%%%%%%%%
\end{document}