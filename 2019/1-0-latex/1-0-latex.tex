% 1-0-warmup.tex

%%%%%%%%%%%%%%%%%%%%
\documentclass[a4paper, justified]{tufte-handout}

%%%%%%%%%%%%%%%%%%%%%%%%%%%%%%%%%%%
% File: hw-preamble.tex
%%%%%%%%%%%%%%%%%%%%%%%%%%%%%%%%%%%

% Set fonts commands
\newcommand{\song}{\CJKfamily{song}} 
\newcommand{\hei}{\CJKfamily{hei}} 
\newcommand{\kai}{\CJKfamily{kai}} 
\newcommand{\fs}{\CJKfamily{fs}}

% colors
\newcommand{\red}[1]{\textcolor{red}{#1}}
\newcommand{\blue}[1]{\textcolor{blue}{#1}}
\newcommand{\teal}[1]{\textcolor{teal}{#1}}

% math
\newcommand{\set}[1]{\{#1\}}

% For math
\usepackage{amsmath}

% Define theorem-like environments
\usepackage[amsmath, thmmarks]{ntheorem}

\theoremstyle{break}
\theorembodyfont{\song}
\theoremseparator{}
\newtheorem*{problem}{Problem}

\theorempreskip{2.0\topsep}
\theoremheaderfont{\kai\bfseries}
\theoremseparator{:}
\newtheorem*{solution}{Solution}

\theoremstyle{break}
\theorempostwork{\bigskip\hrule}
\theoremsymbol{\ensuremath{\Box}}
\newtheorem*{proof}{Proof}

% algorithms
\usepackage[]{algorithm}
\usepackage[noend]{algpseudocode} % noend
% See [Adjust the indentation whithin the algorithmicx-package when a line is broken](https://tex.stackexchange.com/a/68540/23098)
\newcommand{\algparbox}[1]{\parbox[t]{\dimexpr\linewidth-\algorithmicindent}{#1\strut}}
\newcommand{\hStatex}[0]{\vspace{5pt}}
\makeatletter
\newlength{\trianglerightwidth}
\settowidth{\trianglerightwidth}{$\triangleright$~}
\algnewcommand{\LineComment}[1]{\Statex \hskip\ALG@thistlm \(\triangleright\) #1}
\algnewcommand{\LineCommentCont}[1]{\Statex \hskip\ALG@thistlm%
  \parbox[t]{\dimexpr\linewidth-\ALG@thistlm}{\hangindent=\trianglerightwidth \hangafter=1 \strut$\triangleright$ #1\strut}}
\makeatother

% For figures
% for fig with caption: #1: width/size; #2: fig file; #3: fig caption
\newcommand{\fig}[3]{
  \begin{figure}[htp]
    \centering
      \includegraphics[#1]{#2}
      \caption{#3}
  \end{figure}
}

% for fig without caption: #1: width/size; #2: fig file
\newcommand{\fignocaption}[2]{
  \begin{figure}[htp]
    \centering
    \includegraphics[#1]{#2}
  \end{figure}
} % feel free to modify this file
%%%%%%%%%%%%%%%%%%%%
\title{第0讲:\LaTeX}
\me{魏恒峰}{hfwei@nju.edu.cn}{}{}
\date{\zhtoday} % or like 2019年9月13日
%%%%%%%%%%%%%%%%%%%%
\begin{document}
\maketitle
\mfigcap{width = 0.60\linewidth}{figs/knuth-reading-tex-book}{Donald Knuth and \TeX}
\mfigcap{width = 0.60\linewidth}{figs/lamport}{Leslie Lamport for \LaTeX}
%%%%%%%%%%%%%%%%%%%%
\noplagiarism % always keep this line
%%%%%%%%%%%%%%%%%%%%
\begin{abstract}
  \begin{center}{\fcolorbox{blue}{yellow!60}{\parbox{0.30\textwidth}{\large 
    \begin{itemize}
      \item 练习使用 \LaTeX{}
      \item 熟悉如何提交作业
    \end{itemize}}}}
  \end{center}
\end{abstract}
%%%%%%%%%%%%%%%%%%%%
\beginrequired

%%%%%%%%%%%%%%%
\begin{problem}[图片]
  在此处插入~\footnote{方法参见文件 `hw-preamble.tex'}
  一幅你喜欢的图片~\footnote{人物、风景、漫画、海报、艺术$\cdots$}。
\end{problem}

\begin{solution}
\end{solution}
%%%%%%%%%%%%%%%

%%%%%%%%%%%%%%%
\begin{problem}[公式]
  请用\LaTeX{}输出下图中的公式
  ~\footnote{Tool: \href{http://detexify.kirelabs.org/classify.html}{Detexify}}
  ~\footnote{Tool: \href{https://mathpix.com/}{mathpix}}。

  \fig{width = 0.80\textwidth}{figs/formula}
\end{problem}

\begin{solution}
  \[
    \textrm{\textsf{RState}} = \red{\textrm{Your code here}\cdots}
  \]
\end{solution}
%%%%%%%%%%%%%%%

\newpage % to start a new page
%%%%%%%%%%%%%%%
\begin{problem}[表格]
  请用\LaTeX{}输入下图中的表格~\footnote{Tool: \href{http://www.tablesgenerator.com/}{Tables Generator}}。

  \fig{width = 0.15\textwidth}{figs/table}
\end{problem}

\begin{solution}
  \begin{Verbatim}
    \begin{table}[h]
      \centering
    \end{table}
  \end{Verbatim}
\end{solution}
%%%%%%%%%%%%%%%
%%%%%%%%%%%%%%%%%%%%
\beginoptional

%%%%%%%%%%%%%%%
\begin{problem}[算法]
  完成下列算法伪代码~\footnote{Package \href{http://tug.ctan.org/macros/latex/contrib/algorithmicx/algorithmicx.pdf}{algorithmicx}}~\footnote{欢迎贡献代码:\\ \href{https://github.com/hengxin/algorithms-pseudocode}{algorithms-pseudocode@hengxin}}。
\end{problem}

\begin{solution}%
  伪代码如下:
  \begin{algorithm}[H]
    \caption{Sum of integers from 1 to $n$.}
    \label{alg:sum}
    \begin{algorithmic}[1]
      \Procedure{Sum}{$n$}
	\State $sum \gets 0$ \Comment{Initialization}
	\State $\cdots$ \Comment{\red{Replaced with your code}}
	\State \Return  $sum$ \Comment{Return}
      \EndProcedure
    \end{algorithmic}
  \end{algorithm}
\end{solution}
%%%%%%%%%%%%%%%

%%%%%%%%%%%%%%%
\begin{problem}[\ot{} 作图]
  给出绘制下图的 TikZ 代码
  ~\footnote{你需要用到:
    \begin{itemize}
      \item ``foreach''
      \item ``draw''
      \item ``node''
      \item ``bend left'', ``bend right''
    \end{itemize}
  }
  ~\footnote{\href{https://en.wikibooks.org/wiki/LaTeX/PGF/TikZ}{TikZ@wikibooks}}
  ~\footnote{\href{http://mirror.utexas.edu/ctan/graphics/pgf/base/doc/pgfmanual.pdf}
  {pgfmanual ($>1000$ pages)}}。

  \fig{width = 0.80\textwidth}{figs/for-draw}
\end{problem}

\begin{solution}
  \begin{Verbatim}
    \documentclass[tikz]{standalone}
    \begin{document}
      \begin{tikzpicture}[]
      \end{tikzpicture}
    \end{document}
  \end{Verbatim}
\end{solution}
%%%%%%%%%%%%%%%
%%%%%%%%%%%%%%%%%%%%
\beginfb

你可以写~\footnote{优先推荐 \href{http://39.100.120.199}{ProblemOverflow}}:
\begin{itemize}
  \item 对课程及教师的建议与意见
  \item 教材中不理解的内容
  \item 希望深入了解的内容
  \item $\cdots$
\end{itemize}
%%%%%%%%%%%%%%%%%%%%
\end{document}