% 1-2-reasoning.tex

%%%%%%%%%%%%%%%%%%%%
\documentclass[a4paper, justified]{tufte-handout}

%%%%%%%%%%%%%%%%%%%%%%%%%%%%%%%%%%%
% File: hw-preamble.tex
%%%%%%%%%%%%%%%%%%%%%%%%%%%%%%%%%%%

% Set fonts commands
\newcommand{\song}{\CJKfamily{song}} 
\newcommand{\hei}{\CJKfamily{hei}} 
\newcommand{\kai}{\CJKfamily{kai}} 
\newcommand{\fs}{\CJKfamily{fs}}

% colors
\newcommand{\red}[1]{\textcolor{red}{#1}}
\newcommand{\blue}[1]{\textcolor{blue}{#1}}
\newcommand{\teal}[1]{\textcolor{teal}{#1}}

% math
\newcommand{\set}[1]{\{#1\}}

% For math
\usepackage{amsmath}

% Define theorem-like environments
\usepackage[amsmath, thmmarks]{ntheorem}

\theoremstyle{break}
\theorembodyfont{\song}
\theoremseparator{}
\newtheorem*{problem}{Problem}

\theorempreskip{2.0\topsep}
\theoremheaderfont{\kai\bfseries}
\theoremseparator{:}
\newtheorem*{solution}{Solution}

\theoremstyle{break}
\theorempostwork{\bigskip\hrule}
\theoremsymbol{\ensuremath{\Box}}
\newtheorem*{proof}{Proof}

% algorithms
\usepackage[]{algorithm}
\usepackage[noend]{algpseudocode} % noend
% See [Adjust the indentation whithin the algorithmicx-package when a line is broken](https://tex.stackexchange.com/a/68540/23098)
\newcommand{\algparbox}[1]{\parbox[t]{\dimexpr\linewidth-\algorithmicindent}{#1\strut}}
\newcommand{\hStatex}[0]{\vspace{5pt}}
\makeatletter
\newlength{\trianglerightwidth}
\settowidth{\trianglerightwidth}{$\triangleright$~}
\algnewcommand{\LineComment}[1]{\Statex \hskip\ALG@thistlm \(\triangleright\) #1}
\algnewcommand{\LineCommentCont}[1]{\Statex \hskip\ALG@thistlm%
  \parbox[t]{\dimexpr\linewidth-\ALG@thistlm}{\hangindent=\trianglerightwidth \hangafter=1 \strut$\triangleright$ #1\strut}}
\makeatother

% For figures
% for fig with caption: #1: width/size; #2: fig file; #3: fig caption
\newcommand{\fig}[3]{
  \begin{figure}[htp]
    \centering
      \includegraphics[#1]{#2}
      \caption{#3}
  \end{figure}
}

% for fig without caption: #1: width/size; #2: fig file
\newcommand{\fignocaption}[2]{
  \begin{figure}[htp]
    \centering
    \includegraphics[#1]{#2}
  \end{figure}
} % feel free to modify this file
%%%%%%%%%%%%%%%%%%%%
\title{第2讲: 什么样的推理是正确的?}
\me{魏恒峰}{hfwei@nju.edu.cn}{}{}
\date{\zhtoday} % or like 2019年9月13日
%%%%%%%%%%%%%%%%%%%%
\begin{document}
\maketitle
%%%%%%%%%%%%%%%%%%%%
\noplagiarism % always keep this line
%%%%%%%%%%%%%%%%%%%%
\begin{abstract}
  \begin{center}{\fcolorbox{blue}{yellow!60}{\parbox{0.40\textwidth}{\large 
    \begin{itemize}
      \item 消除对``符号''的恐惧
      \item 培养与``逻辑''的亲密情感
    \end{itemize}}}}
  \end{center}
\end{abstract}
%%%%%%%%%%%%%%%%%%%%
\beginrequired

%%%%%%%%%%%%%%%
\begin{problem}[改编自 UD Exercise $2.1$]
  以下哪些是命题? 请简要说明理由。

  \begin{enumerate}[(1)]
    \item $X + 6 = 0$
    \item $X = X$
    \item 哥德巴赫猜想
    \item 今天是雨天
    \item 明天是晴天
    \item 明天是周二
    \item 这句话是假话
  \end{enumerate}
\end{problem}

\begin{solution}
\end{solution}
%%%%%%%%%%%%%%%

%%%%%%%%%%%%%%%
\begin{problem}[关于笛卡尔的一则笑话]
  \mfigcap{width = 0.60\textwidth}{figs/Descartes}{Ren\'e Descartes (1596 $\sim$ 1650)}
  笛卡尔是法国著名哲学家、物理学家、数学家、神学家。
  有一天,他走进一家酒吧。
  酒吧服务员问,``要来一杯吗?''。
  笛卡尔说,``I think not''~\footnote{嗯,在这道题里,笛卡尔讲英语。}。
  话音刚落,笛卡尔消失了。

  \begin{enumerate}[(1)]
    \item 请问,这则笑话的笑点在哪~\footnote{想想笛卡尔说过什么 (英文版本)?}?
    \item 请问,这则笑话在逻辑上是否有漏洞?
  \end{enumerate}
\end{problem}

\begin{solution}
\end{solution}
%%%%%%%%%%%%%%%

%%%%%%%%%%%%%%%
\begin{problem}[UD Problem $2.5$]
\end{problem}

\begin{solution}
\end{solution}
%%%%%%%%%%%%%%%

%%%%%%%%%%%%%%%
\begin{problem}[UD Problem $2.7\; (a, c, f)$]
\end{problem}

\begin{solution}
\end{solution}
%%%%%%%%%%%%%%%

%%%%%%%%%%%%%%%
\begin{problem}[UD Problem $2.16$]
\end{problem}

\begin{solution}
\end{solution}
%%%%%%%%%%%%%%%

%%%%%%%%%%%%%%%
\begin{problem}[UD Problem $2.18$]
\end{problem}

\begin{solution}
\end{solution}
%%%%%%%%%%%%%%%

%%%%%%%%%%%%%%%
\begin{problem}[UD Problem $3.3\; (d)$]
\end{problem}

\begin{solution}
\end{solution}
%%%%%%%%%%%%%%%

%%%%%%%%%%%%%%%
\begin{problem}[UD Problem $3.10$]
\end{problem}

\begin{solution}
\end{solution}
%%%%%%%%%%%%%%%

%%%%%%%%%%%%%%%
\begin{problem}[UD Problem $3.12$]
\end{problem}

\begin{solution}
\end{solution}
%%%%%%%%%%%%%%%

%%%%%%%%%%%%%%%
\begin{problem}[UD Problem $4.1$]
\end{problem}

\begin{solution}
\end{solution}
%%%%%%%%%%%%%%%

%%%%%%%%%%%%%%%
\begin{problem}[UD Problem $4.5\; (j, k)$]
\end{problem}

\begin{solution}
\end{solution}
%%%%%%%%%%%%%%%

%%%%%%%%%%%%%%%
\begin{problem}[两种连续性]
  A function $f$ from $\mathbb{R}$ to $\mathbb{R}$ is called 
  \begin{itemize}
    \item \emph{pointwise continuous} if
      for every $x \in \mathbb{R}$ 
      and every real number $\epsilon > 0$,
      there exists real $\delta > 0$ such that 
      for every $y \in \mathbb{R}$ with $|x - y| < \delta$, 
      we have that $|f(x) -  f(y)|< \epsilon$. 
    \item \emph{uniformly continuous} if
      for every real number $\epsilon > 0$,
      there exists real $\delta > 0$ such that 
      for every $x, y \in \mathbb{R}$ with $|x - y| < \delta$, 
      we have that $|f(x) -  f(y)|< \epsilon$. 
  \end{itemize}

  \begin{enumerate}[(1)]
    \item 请用一阶谓词逻辑公式表示上述定义。
    \item 请比较两种连续性的``强弱''关系,并举例说明。
  \end{enumerate}
\end{problem}

\begin{solution}
\end{solution}
%%%%%%%%%%%%%%%

%%%%%%%%%%%%%%%
\begin{problem}[UD Problem $4.9\; (a, c)$]
\end{problem}

\begin{solution}
\end{solution}
%%%%%%%%%%%%%%%

%%%%%%%%%%%%%%%
\begin{problem}[UD Problem $4.20$]
\end{problem}

\begin{solution}
\end{solution}
%%%%%%%%%%%%%%%

%%%%%%%%%%%%%%%%%%%%
\beginoptional
%%%%%%%%%%%%%%%
\begin{problem}[Hilbert 式的命题逻辑推理系统]
  \mfigcap{width = 0.55\textwidth}{figs/Hilbert}{David Hilbert (1862 $\sim$ 1943)}
  我们平常使用的推理系统是自然推理系统。
  本题介绍另一种推理系统,称为 Hilbert 式的推理系统。
  它的特点是有多条公理,但只有一条推理规则,而且推理是线性的。
  对于本题而言,我们只需要使用其中两条公理 
  (其中, $\alpha, \beta, \gamma$ 为任意命题):
  \begin{enumerate}[(a)]
    \item $\alpha \to (\beta \to \alpha)$
    \item $\big(\alpha \to (\beta \to \gamma)\big) \to 
      \big((\alpha \to \beta) \to (\alpha \to \gamma) \big)$
    \end{enumerate}
  推理规则是: 从 $\alpha$ 与 $\alpha \to \beta$, 可以推出 $\beta$。

  \vspace{8pt}
  \noindent 问题: 请在上述公理系统内~\footnote{这意味着,你能且仅能使用该系统中规定的公理以及推理规则。}
  证明 $\alpha \to \alpha$。
\end{problem}

\begin{solution}
\end{solution}
%%%%%%%%%%%%%%%

%%%%%%%%%%%%%%%%%%%%
\beginot
%%%%%%%%%%%%%%%
\begin{ot}[自然推理系统]
  \mfigcap{width = 0.55\textwidth}{figs/Gentzen}{Gerhard Gentzen (1909 $\sim$ 1945)}
  请结合 Coq 
  \href{https://github.com/hengxin/problem-solving-class-coq/blob/master/2019-1-coq/Logic.v}{Logic.v}
  介绍一阶谓词逻辑(包含命题逻辑)的自然推理系统 (Designed by Gerhard Gentzen)。

  参考资料:
  \begin{itemize}
    \item \href{https://github.com/hengxin/problem-solving-class-coq/blob/master/2019-1-coq/Logic.v}{Logic.v} in Coq
    \item \href{https://www.cs.cornell.edu/courses/cs3110/2013sp/lectures/lec15-logic-contd/lec15.html}{Natural Deduction for Propositional Logic @ cs.cornell.edu}
    \item \href{http://leanprover.github.io/logic\_and\_proof/natural\_deduction\_for\_propositional\_logic.html}
      {Natural Deduction for Propositional Logic @ leanprover.github.io}
  \end{itemize}
\end{ot}

\begin{solution}
\end{solution}
%%%%%%%%%%%%%%%

%%%%%%%%%%%%%%%
\begin{ot}[前束范式]
  介绍一阶谓词逻辑中的前束范式 (Prenex Normal Form), 包括但不限于:
  \begin{itemize}
    \item 定义
    \item 转换方法与举例
    \item 用途简介
  \end{itemize}

  参考资料:
  \begin{itemize}
    \item \href{https://en.wikipedia.org/wiki/Prenex\_normal\_form}{Prenex normal form @ wiki}
  \end{itemize}
\end{ot}

\begin{solution}
\end{solution}
%%%%%%%%%%%%%%%

%%%%%%%%%%%%%%%%%%%%
\begincorrection
%%%%%%%%%%%%%%%%%%%%

%%%%%%%%%%%%%%%%%%%%
\beginfb

你可以写
~\footnote{优先推荐 \href{problemoverflow.top}{ProblemOverflow}}:
\begin{itemize}
  \item 对课程及教师的建议与意见
  \item 教材中不理解的内容
  \item 希望深入了解的内容
  \item $\cdots$
\end{itemize}
%%%%%%%%%%%%%%%%%%%%
\end{document}