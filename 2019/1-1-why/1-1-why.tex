% 1-1-why.tex

%%%%%%%%%%%%%%%%%%%%
\documentclass[a4paper, justified]{tufte-handout}

%%%%%%%%%%%%%%%%%%%%%%%%%%%%%%%%%%%
% File: hw-preamble.tex
%%%%%%%%%%%%%%%%%%%%%%%%%%%%%%%%%%%

% Set fonts commands
\newcommand{\song}{\CJKfamily{song}} 
\newcommand{\hei}{\CJKfamily{hei}} 
\newcommand{\kai}{\CJKfamily{kai}} 
\newcommand{\fs}{\CJKfamily{fs}}

% colors
\newcommand{\red}[1]{\textcolor{red}{#1}}
\newcommand{\blue}[1]{\textcolor{blue}{#1}}
\newcommand{\teal}[1]{\textcolor{teal}{#1}}

% math
\newcommand{\set}[1]{\{#1\}}

% For math
\usepackage{amsmath}

% Define theorem-like environments
\usepackage[amsmath, thmmarks]{ntheorem}

\theoremstyle{break}
\theorembodyfont{\song}
\theoremseparator{}
\newtheorem*{problem}{Problem}

\theorempreskip{2.0\topsep}
\theoremheaderfont{\kai\bfseries}
\theoremseparator{:}
\newtheorem*{solution}{Solution}

\theoremstyle{break}
\theorempostwork{\bigskip\hrule}
\theoremsymbol{\ensuremath{\Box}}
\newtheorem*{proof}{Proof}

% algorithms
\usepackage[]{algorithm}
\usepackage[noend]{algpseudocode} % noend
% See [Adjust the indentation whithin the algorithmicx-package when a line is broken](https://tex.stackexchange.com/a/68540/23098)
\newcommand{\algparbox}[1]{\parbox[t]{\dimexpr\linewidth-\algorithmicindent}{#1\strut}}
\newcommand{\hStatex}[0]{\vspace{5pt}}
\makeatletter
\newlength{\trianglerightwidth}
\settowidth{\trianglerightwidth}{$\triangleright$~}
\algnewcommand{\LineComment}[1]{\Statex \hskip\ALG@thistlm \(\triangleright\) #1}
\algnewcommand{\LineCommentCont}[1]{\Statex \hskip\ALG@thistlm%
  \parbox[t]{\dimexpr\linewidth-\ALG@thistlm}{\hangindent=\trianglerightwidth \hangafter=1 \strut$\triangleright$ #1\strut}}
\makeatother

% For figures
% for fig with caption: #1: width/size; #2: fig file; #3: fig caption
\newcommand{\fig}[3]{
  \begin{figure}[htp]
    \centering
      \includegraphics[#1]{#2}
      \caption{#3}
  \end{figure}
}

% for fig without caption: #1: width/size; #2: fig file
\newcommand{\fignocaption}[2]{
  \begin{figure}[htp]
    \centering
    \includegraphics[#1]{#2}
  \end{figure}
} % feel free to modify this file
%%%%%%%%%%%%%%%%%%%%
\title{第1讲: 为什么计算机能解题?}
\me{魏恒峰}{hfwei@nju.edu.cn}{}{}
\date{\zhtoday} % or like 2019年9月13日
%%%%%%%%%%%%%%%%%%%%
\begin{document}
\maketitle
%%%%%%%%%%%%%%%%%%%%
\noplagiarism % always keep this line
%%%%%%%%%%%%%%%%%%%%
\begin{abstract}
  \mfig{width = 0.60\linewidth}{figs/recursion-mirror}
  \begin{center}{\fcolorbox{blue}{yellow!60}{\parbox{0.50\textwidth}{\large 
    \begin{itemize}
      \item 体会``思维的乐趣''
      \item 初步了解递归与数学归纳法 
      \item 初步接触算法概念与问题下界概念
    \end{itemize}}}}
  \end{center}
\end{abstract}
%%%%%%%%%%%%%%%%%%%%
\beginrequired

%%%%%%%%%%%%%%%
\begin{problem}[UD Problem $1.6$]
  The following message is encoded using a shifted alphabet just as in Exercise 1.1. 
  What does the message say?  
  
  RDSXCVIWTDGNXHUJCLTLXAAATPGCBDGTPQDJIXIAPITG

  请简单描述你的解题思路。
\end{problem}

\begin{solution}
\end{solution}
%%%%%%%%%%%%%%%

%%%%%%%%%%%%%%%
\begin{problem}[UD Problem $1.9$]
  Let $n$ be an odd integer. Prove that $n^3 − n$ is divisible by 24.
\end{problem}

\begin{solution}
\end{solution}
%%%%%%%%%%%%%%%

%%%%%%%%%%%%%%%
\begin{problem}[$n$ 枚硬币]
  \mfig{width = 0.70\linewidth}{figs/scale}

  你有 $n$ 枚外观一模一样的硬币。
  已知其中有一枚假币,并且假币的质量比真币轻。\\
  现有一个带两个托盘的天平秤。
  请设计\red{``称量''}~\footnote{只允许使用``称量''操作。这是我们在做算法分析时关注的关键操作。}\red{方案}~\footnote{这就是算法},找到这枚假币。

  请用尽可能简洁的自然语言或者伪代码描述你的称量方案。
  不要提交可执行代码。
\end{problem}

\begin{solution}
\end{solution}
%%%%%%%%%%%%%%%

%%%%%%%%%%%%%%%
\begin{problem}[$n$ 枚硬币问题的下界]
  接上一题,
  \red{最少}
  ~\footnote{这就是问题的下界。显然,只考虑特定的\blue{\it 算法}是不够的;你要考虑\blue{\it 问题}本身的性质以及``称量''操作的本质。}
  需要称量多少次,才能找到这枚假币?
  请证明你的结论。
\end{problem}

\begin{solution}
\end{solution}
%%%%%%%%%%%%%%%

%%%%%%%%%%%%%%%
\begin{problem}[$12$ 枚硬币 (UD Problem $1.8$)]
  你有 $12$ 枚外观一模一样的硬币。
  已知其中有一枚假币,其质量与真币不同。\\
  \red{但是,你不知道假币比真币轻还是重}。
  只称量三次,如何找出这枚假币,并确定它相对于真币的轻重?

  \mfig{width = 0.80\textwidth}{figs/try}
\end{problem}

\begin{solution}%
\end{solution}
%%%%%%%%%%%%%%%

%%%%%%%%%%%%%%%%%%%%
\beginoptional
%%%%%%%%%%%%%%%
\begin{problem}[$n$ 枚硬币]
  你有 $n$ 枚外观一模一样的硬币。
  已知其中有一枚假币,其质量与真币不同。\\
  但是,你不知道假币比真币轻还是重。
  \red{好在,每个硬币都有一个标签 ``Possibly Heavy'' 或者 ``Possibly Light''。}
  请设计``称量''方案,找出这枚假币,并确定它相对于真币的轻重。
\end{problem}

\begin{solution}
\end{solution}
%%%%%%%%%%%%%%%
\begin{problem}[\ot{} $n$ 枚硬币]
  你有 $n$ 枚外观一模一样的硬币。
  已知其中有一枚假币,其质量与真币不同。\\
  \red{但是,你不知道假币比真币轻还是重,硬币上也没有标签}。
  请用尽可能少~\footnote{``称量''会带来什么信息?这些信息会如何影响问题的性质?}的称量次数,
  找到这枚假币并确定它相对于真币的轻重。
\end{problem}

\begin{solution}
\end{solution}
%%%%%%%%%%%%%%%
%%%%%%%%%%%%%%%%%%%%
\beginfb

你可以写
~\footnote{优先推荐 \href{http://39.100.120.199}{ProblemOverflow}}:
\begin{itemize}
  \item 对课程及教师的建议与意见
  \item 教材中不理解的内容
  \item 希望深入了解的内容
  \item $\cdots$
\end{itemize}
%%%%%%%%%%%%%%%%%%%%
\end{document}