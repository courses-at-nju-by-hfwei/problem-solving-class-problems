% 2-2-efficiency.tex

%%%%%%%%%%%%%%%%%%%%
\documentclass[a4paper, justified]{tufte-handout}

%%%%%%%%%%%%%%%%%%%%%%%%%%%%%%%%%%%
% File: hw-preamble.tex
%%%%%%%%%%%%%%%%%%%%%%%%%%%%%%%%%%%

% Set fonts commands
\newcommand{\song}{\CJKfamily{song}} 
\newcommand{\hei}{\CJKfamily{hei}} 
\newcommand{\kai}{\CJKfamily{kai}} 
\newcommand{\fs}{\CJKfamily{fs}}

% colors
\newcommand{\red}[1]{\textcolor{red}{#1}}
\newcommand{\blue}[1]{\textcolor{blue}{#1}}
\newcommand{\teal}[1]{\textcolor{teal}{#1}}

% math
\newcommand{\set}[1]{\{#1\}}

% For math
\usepackage{amsmath}

% Define theorem-like environments
\usepackage[amsmath, thmmarks]{ntheorem}

\theoremstyle{break}
\theorembodyfont{\song}
\theoremseparator{}
\newtheorem*{problem}{Problem}

\theorempreskip{2.0\topsep}
\theoremheaderfont{\kai\bfseries}
\theoremseparator{:}
\newtheorem*{solution}{Solution}

\theoremstyle{break}
\theorempostwork{\bigskip\hrule}
\theoremsymbol{\ensuremath{\Box}}
\newtheorem*{proof}{Proof}

% algorithms
\usepackage[]{algorithm}
\usepackage[noend]{algpseudocode} % noend
% See [Adjust the indentation whithin the algorithmicx-package when a line is broken](https://tex.stackexchange.com/a/68540/23098)
\newcommand{\algparbox}[1]{\parbox[t]{\dimexpr\linewidth-\algorithmicindent}{#1\strut}}
\newcommand{\hStatex}[0]{\vspace{5pt}}
\makeatletter
\newlength{\trianglerightwidth}
\settowidth{\trianglerightwidth}{$\triangleright$~}
\algnewcommand{\LineComment}[1]{\Statex \hskip\ALG@thistlm \(\triangleright\) #1}
\algnewcommand{\LineCommentCont}[1]{\Statex \hskip\ALG@thistlm%
  \parbox[t]{\dimexpr\linewidth-\ALG@thistlm}{\hangindent=\trianglerightwidth \hangafter=1 \strut$\triangleright$ #1\strut}}
\makeatother

% For figures
% for fig with caption: #1: width/size; #2: fig file; #3: fig caption
\newcommand{\fig}[3]{
  \begin{figure}[htp]
    \centering
      \includegraphics[#1]{#2}
      \caption{#3}
  \end{figure}
}

% for fig without caption: #1: width/size; #2: fig file
\newcommand{\fignocaption}[2]{
  \begin{figure}[htp]
    \centering
    \includegraphics[#1]{#2}
  \end{figure}
} % feel free to modify this file
%%%%%%%%%%%%%%%%%%%%
\title{第2讲: 算法的效率}
\me{魏恒峰}{hfwei@nju.edu.cn}{}{}
\date{\zhtoday} % or like 2019年9月13日
%%%%%%%%%%%%%%%%%%%%
\begin{document}
\maketitle
%%%%%%%%%%%%%%%%%%%%
\noplagiarism % always keep this line
%%%%%%%%%%%%%%%%%%%%
\begin{abstract}
  % \mfigcap{width = 0.85\textwidth}{figs/George-Boole}{George Boole}
  % \begin{center}{\fcolorbox{blue}{yellow!60}{\parbox{0.65\textwidth}{\large 
  %   \begin{itemize}
  %     \item 
  %   \end{itemize}}}}
  % \end{center}
\end{abstract}
%%%%%%%%%%%%%%%%%%%%
\beginrequired

%%%%%%%%%%%%%%%
\begin{problem}[DH Problem 6.18: $\log_{m} n$]
\end{problem}

\begin{solution}
\end{solution}
%%%%%%%%%%%%%%%

%%%%%%%%%%%%%%%
\begin{problem}[DH Problem 6.19: $\log_{m} n$]
\end{problem}

\begin{solution}
\end{solution}
%%%%%%%%%%%%%%%

%%%%%%%%%%%%%%%
\begin{problem}[DH Problem 6.20 (a): $\log_{m} n$]
\end{problem}

\begin{solution}
\end{solution}
%%%%%%%%%%%%%%%

%%%%%%%%%%%%%%%
\begin{problem}[TC Exercise 3.1-6]
\end{problem}

\begin{solution}
\end{solution}
%%%%%%%%%%%%%%%

%%%%%%%%%%%%%%%
\begin{problem}[TC Exercise 3.1-7]
\end{problem}

\begin{solution}
\end{solution}
%%%%%%%%%%%%%%%

%%%%%%%%%%%%%%%
\begin{problem}[TC Problem 3-3 (a)]
\end{problem}

\begin{solution}
\end{solution}
%%%%%%%%%%%%%%%

%%%%%%%%%%%%%%%%%%%%
\beginoptional

%%%%%%%%%%%%%%%
\begin{problem}[DH Problem 6.13]
\end{problem}

\begin{solution}
\end{solution}
%%%%%%%%%%%%%%%

%%%%%%%%%%%%%%%%%%%%
\beginot

%%%%%%%%%%%%%%%
本次 OT 介绍两种证明问题下界的常用技术。

\begin{ot}[Decision Trees]
  介绍 Decison Trees (决策树) 的概念以及在证明问题下界时的应用
  (包括但不限于本次选做题 DH 6.13)。

  参考资料:
  \begin{itemize}
    \item \href{https://en.wikipedia.org/wiki/Decision\_tree\_model}{Decision tree model @ wiki}
    \item \href{http://jeffe.cs.illinois.edu/teaching/algorithms/notes/12-lowerbounds.pdf}{lecture-note by jeffe}
  \end{itemize}
\end{ot}

% \begin{solution}
% \end{solution}
%%%%%%%%%%%%%%%
\vspace{0.50cm}
%%%%%%%%%%%%%%%
\begin{ot}[Adversary Argument]
  介绍 Adversary Argument (对手论证) 的概念
  以及在证明问题下界时的应用。

  \begin{itemize}
    \item \href{http://jeffe.cs.illinois.edu/teaching/algorithms/notes/13-adversary.pdf}{lecture-note by jeffe}
  \end{itemize}
\end{ot}

% \begin{solution}
% \end{solution}
%%%%%%%%%%%%%%%

%%%%%%%%%%%%%%%%%%%%
% 如果没有需要订正的题目,可以把这部分删掉

% \begincorrection
%%%%%%%%%%%%%%%%%%%%

%%%%%%%%%%%%%%%%%%%%
% 如果没有反馈,可以把这部分删掉
\beginfb

% 你可以写
% ~\footnote{优先推荐 \href{problemoverflow.top}{ProblemOverflow}}:
% \begin{itemize}
%   \item 对课程及教师的建议与意见
%   \item 教材中不理解的内容
%   \item 希望深入了解的内容
%   \item $\cdots$
% \end{itemize}
%%%%%%%%%%%%%%%%%%%%
\end{document}