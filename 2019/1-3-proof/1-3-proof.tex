% 1-3-proof.tex

%%%%%%%%%%%%%%%%%%%%
\documentclass[a4paper, justified]{tufte-handout}

%%%%%%%%%%%%%%%%%%%%%%%%%%%%%%%%%%%
% File: hw-preamble.tex
%%%%%%%%%%%%%%%%%%%%%%%%%%%%%%%%%%%

% Set fonts commands
\newcommand{\song}{\CJKfamily{song}} 
\newcommand{\hei}{\CJKfamily{hei}} 
\newcommand{\kai}{\CJKfamily{kai}} 
\newcommand{\fs}{\CJKfamily{fs}}

% colors
\newcommand{\red}[1]{\textcolor{red}{#1}}
\newcommand{\blue}[1]{\textcolor{blue}{#1}}
\newcommand{\teal}[1]{\textcolor{teal}{#1}}

% math
\newcommand{\set}[1]{\{#1\}}

% For math
\usepackage{amsmath}

% Define theorem-like environments
\usepackage[amsmath, thmmarks]{ntheorem}

\theoremstyle{break}
\theorembodyfont{\song}
\theoremseparator{}
\newtheorem*{problem}{Problem}

\theorempreskip{2.0\topsep}
\theoremheaderfont{\kai\bfseries}
\theoremseparator{:}
\newtheorem*{solution}{Solution}

\theoremstyle{break}
\theorempostwork{\bigskip\hrule}
\theoremsymbol{\ensuremath{\Box}}
\newtheorem*{proof}{Proof}

% algorithms
\usepackage[]{algorithm}
\usepackage[noend]{algpseudocode} % noend
% See [Adjust the indentation whithin the algorithmicx-package when a line is broken](https://tex.stackexchange.com/a/68540/23098)
\newcommand{\algparbox}[1]{\parbox[t]{\dimexpr\linewidth-\algorithmicindent}{#1\strut}}
\newcommand{\hStatex}[0]{\vspace{5pt}}
\makeatletter
\newlength{\trianglerightwidth}
\settowidth{\trianglerightwidth}{$\triangleright$~}
\algnewcommand{\LineComment}[1]{\Statex \hskip\ALG@thistlm \(\triangleright\) #1}
\algnewcommand{\LineCommentCont}[1]{\Statex \hskip\ALG@thistlm%
  \parbox[t]{\dimexpr\linewidth-\ALG@thistlm}{\hangindent=\trianglerightwidth \hangafter=1 \strut$\triangleright$ #1\strut}}
\makeatother

% For figures
% for fig with caption: #1: width/size; #2: fig file; #3: fig caption
\newcommand{\fig}[3]{
  \begin{figure}[htp]
    \centering
      \includegraphics[#1]{#2}
      \caption{#3}
  \end{figure}
}

% for fig without caption: #1: width/size; #2: fig file
\newcommand{\fignocaption}[2]{
  \begin{figure}[htp]
    \centering
    \includegraphics[#1]{#2}
  \end{figure}
} % feel free to modify this file
%%%%%%%%%%%%%%%%%%%%
\title{第3讲: 常用的证明方法}
\me{魏恒峰}{hfwei@nju.edu.cn}{}{}
\date{\zhtoday} % or like 2019年9月13日
%%%%%%%%%%%%%%%%%%%%
\begin{document}
\maketitle
%%%%%%%%%%%%%%%%%%%%
\noplagiarism % always keep this line
%%%%%%%%%%%%%%%%%%%%
\begin{abstract}
  \mfig{width = 0.95\textwidth}{figs/domino}
  \begin{center}{\fcolorbox{blue}{yellow!60}{\parbox{0.58\textwidth}{\large 
    \begin{itemize}
      \item 数学归纳法是你最好的朋友
      \item 反证法也是你最好的朋友
      \item 鸽笼原理, 哦, 有点高冷, 这个朋友不好交
    \end{itemize}}}}
  \end{center}
\end{abstract}
%%%%%%%%%%%%%%%%%%%%
\beginrequired

%%%%%%%%%%%%%%%
\begin{problem}[UD Problem $5.12$: $3k + 2$]
\end{problem}

\begin{solution}
\end{solution}
%%%%%%%%%%%%%%%

%%%%%%%%%%%%%%%
\begin{problem}[UD Problem $5.24$: Squaring]
\end{problem}

\begin{solution}
\end{solution}
%%%%%%%%%%%%%%%

%%%%%%%%%%%%%%%
\begin{problem}[Primes 3 (Mod 4) Theorem]
  请证明: There are infinitely many primes 
  that are congruent to 3 modulo 4.
\end{problem}

\begin{solution}
\end{solution}
%%%%%%%%%%%%%%%

%%%%%%%%%%%%%%%
\begin{problem}[改编自 UD Problem $18.20$ 与 UD Problem $18.26$]
  请证明: 
  \begin{enumerate}[(1)]
    \item ``The first principle of mathematical induction'' (Theorem $18.1$)
      与 ``The second principle of mathematical induction'' (Theorem $18.9$) 等价。
    \item ``The second principle of mathematical induction'' 蕴含 
      ``Well-ordering principles of the natural numbers'' (in Chapter 12)。
  \end{enumerate}
\end{problem}

\begin{solution}
\end{solution}
%%%%%%%%%%%%%%%

%%%%%%%%%%%%%%%
\begin{problem}[UD Problem $18.25$\; (c, d): Binomial]
\end{problem}

\begin{solution}
\end{solution}
%%%%%%%%%%%%%%%

%%%%%%%%%%%%%%%
\begin{problem}[Lines in the Plane]
  \begin{enumerate}[(1)]
    \item What is the maximum number $L_n$ of regions 
      determined by $n$ straight lines in the plane?
      \mfigcap{width = 1.00\textwidth}{figs/straight-line-ln}{Examples for $L_0$, $L_1$, and $L_2$.}
      (注: 直线两端可以无限延长)
    \item What is the maximum number $Z_n$ of regions 
      determined by $n$ bent lines, each containing one ``zig'', 
      in the plane?
      \mfigcap{width = 1.00\textwidth}{figs/bent-line-zn}{Examples for $Z_1$ and $Z_2$.}
      (注: 两端可以无限延长)
    \item What's the maximum number $ZZ_n$ of regions
      determined by $n$ ``zig-zag'' lines in the plane?
      \mfigcap{width = 1.00\textwidth}{figs/zigzag-zzn}{Example for $ZZ_2$.}
      (注: 两端可以无限延长)
  \end{enumerate}
\end{problem}

\begin{solution}
\end{solution}
%%%%%%%%%%%%%%%

%%%%%%%%%%%%%%%
\begin{problem}[ES Problem $24.4$: Distance in Square]
\end{problem}

\begin{solution}
\end{solution}
%%%%%%%%%%%%%%%

%%%%%%%%%%%%%%%
\begin{problem}[ES Problem $24.6$: Lattice Points]
\end{problem}

\begin{solution}
\end{solution}
%%%%%%%%%%%%%%%

%%%%%%%%%%%%%%%
\begin{problem}[ES Problem $24.7$: Monotone Subsequence]
\end{problem}

\begin{solution}
\end{solution}
%%%%%%%%%%%%%%%

%%%%%%%%%%%%%%%%%%%%
\beginoptional

%%%%%%%%%%%%%%%
\begin{problem}[Numbers]
  Suppose $A \subseteq \set{1, 2, \cdots, 2n}$ with $|A| = n + 1$.
  Please prove that:
  \mfig{width = 0.60\textwidth}{figs/pigeon-hole-principle}
  \begin{enumerate}[(1)]
    \item There are two numbers in $A$ which are relatively prime (互素).
    \item There are two numbers in $A$ such that one divides (整除) the other.
  \end{enumerate}
\end{problem}

\begin{solution}
\end{solution}
%%%%%%%%%%%%%%%

%%%%%%%%%%%%%%%%%%%%
\beginot

%%%%%%%%%%%%%%%
\begin{ot}[Coq]
  请介绍如何在 Coq 中使用数学归纳法。

  \noindent 参考资料:
  \begin{itemize}
    \item \href{https://github.com/hengxin/problem-solving-class-coq/blob/master/2019-1-coq/Induction.v}{Induction.v}
  \end{itemize}
\end{ot}

\begin{solution}
\end{solution}
%%%%%%%%%%%%%%%

%%%%%%%%%%%%%%%
\begin{ot}[Double Counting]
  ``Double Counting'' 是一种神奇、漂亮的组合证明技巧。
  请了解 Double Counting 并以 ``Counting Trees'' 为例介绍这种证明技巧。

  \noindent 参考资料:
  \mfig{width = 0.95\textwidth}{figs/good-will-hunting-counting-trees}
  \begin{itemize}
    \item 电影 ``Good Will Hunting'' (心灵捕手)
    \item Chapter 30 ``Cayley's formula for the number of trees''
      of ``Proofs from THE BOOK'' (Fourth Edition)
    \item \href{https://en.wikipedia.org/wiki/Double\_counting\_(proof\_technique)#Counting\_trees}{Counting trees @ wiki}
  \end{itemize}
\end{ot}

\begin{solution}
\end{solution}
%%%%%%%%%%%%%%%

%%%%%%%%%%%%%%%%%%%%
\begincorrection
%%%%%%%%%%%%%%%%%%%%

%%%%%%%%%%%%%%%%%%%%
\beginfb

你可以写
~\footnote{优先推荐 \href{problemoverflow.top}{ProblemOverflow}}:
\begin{itemize}
  \item 对课程及教师的建议与意见
  \item 教材中不理解的内容
  \item 希望深入了解的内容
  \item $\cdots$
\end{itemize}
%%%%%%%%%%%%%%%%%%%%
\end{document}