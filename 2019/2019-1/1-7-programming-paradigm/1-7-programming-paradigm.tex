% 1-7-programming-paradigm.tex

%%%%%%%%%%%%%%%%%%%%
\documentclass[a4paper, justified]{tufte-handout}

%%%%%%%%%%%%%%%%%%%%%%%%%%%%%%%%%%%
% File: hw-preamble.tex
%%%%%%%%%%%%%%%%%%%%%%%%%%%%%%%%%%%

% Set fonts commands
\newcommand{\song}{\CJKfamily{song}} 
\newcommand{\hei}{\CJKfamily{hei}} 
\newcommand{\kai}{\CJKfamily{kai}} 
\newcommand{\fs}{\CJKfamily{fs}}

% colors
\newcommand{\red}[1]{\textcolor{red}{#1}}
\newcommand{\blue}[1]{\textcolor{blue}{#1}}
\newcommand{\teal}[1]{\textcolor{teal}{#1}}

% math
\newcommand{\set}[1]{\{#1\}}

% For math
\usepackage{amsmath}

% Define theorem-like environments
\usepackage[amsmath, thmmarks]{ntheorem}

\theoremstyle{break}
\theorembodyfont{\song}
\theoremseparator{}
\newtheorem*{problem}{Problem}

\theorempreskip{2.0\topsep}
\theoremheaderfont{\kai\bfseries}
\theoremseparator{:}
\newtheorem*{solution}{Solution}

\theoremstyle{break}
\theorempostwork{\bigskip\hrule}
\theoremsymbol{\ensuremath{\Box}}
\newtheorem*{proof}{Proof}

% algorithms
\usepackage[]{algorithm}
\usepackage[noend]{algpseudocode} % noend
% See [Adjust the indentation whithin the algorithmicx-package when a line is broken](https://tex.stackexchange.com/a/68540/23098)
\newcommand{\algparbox}[1]{\parbox[t]{\dimexpr\linewidth-\algorithmicindent}{#1\strut}}
\newcommand{\hStatex}[0]{\vspace{5pt}}
\makeatletter
\newlength{\trianglerightwidth}
\settowidth{\trianglerightwidth}{$\triangleright$~}
\algnewcommand{\LineComment}[1]{\Statex \hskip\ALG@thistlm \(\triangleright\) #1}
\algnewcommand{\LineCommentCont}[1]{\Statex \hskip\ALG@thistlm%
  \parbox[t]{\dimexpr\linewidth-\ALG@thistlm}{\hangindent=\trianglerightwidth \hangafter=1 \strut$\triangleright$ #1\strut}}
\makeatother

% For figures
% for fig with caption: #1: width/size; #2: fig file; #3: fig caption
\newcommand{\fig}[3]{
  \begin{figure}[htp]
    \centering
      \includegraphics[#1]{#2}
      \caption{#3}
  \end{figure}
}

% for fig without caption: #1: width/size; #2: fig file
\newcommand{\fignocaption}[2]{
  \begin{figure}[htp]
    \centering
    \includegraphics[#1]{#2}
  \end{figure}
} % feel free to modify this file
%%%%%%%%%%%%%%%%%%%%
\title{第7讲: 程序设计范型}
\me{魏恒峰}{hfwei@nju.edu.cn}{}{}
\date{\zhtoday} % or like 2019年9月13日
%%%%%%%%%%%%%%%%%%%%
\begin{document}
\maketitle
%%%%%%%%%%%%%%%%%%%%
\noplagiarism % always keep this line
%%%%%%%%%%%%%%%%%%%%
\begin{abstract}
  \mfig{width = 0.80\textwidth}{figs/haskell-side-effects}
  \begin{center}{\fcolorbox{blue}{yellow!60}{\parbox{0.45\textwidth}{\large 
    \begin{itemize}
      \item 函数式程序设计, 你值得拥有
    \end{itemize}}}}
  \end{center}
\end{abstract}
%%%%%%%%%%%%%%%%%%%%
\beginrequired

%%%%%%%%%%%%%%%
\begin{problem}[Coq ``Lists.v'']
  完成 \href{https://github.com/hengxin/problem-solving-class-coq/blob/master/2019-1-coq/Lists.v}{``Lists.v''}。
\end{problem}

\begin{solution}
  请将完成的 ``Lists.v'' (以及它所依赖的 ``.v'' 文件) 与 ``1-7-programming-paradigm.pdf'' 打包提交。
  压缩包命名: 1-7-programming-paradigm-姓名.zip/rar。
\end{solution}
%%%%%%%%%%%%%%%

%%%%%%%%%%%%%%%%%%%%
\beginoptional

%%%%%%%%%%%%%%%
\begin{problem}[Coq ``FP.v'']
  完成 \href{https://github.com/hengxin/problem-solving-class-coq/blob/master/2019-1-coq/FP.v}{``FP.v''} 
  ~\footnote{其实, 这才是我们本节的知识点。}(本节最后的选做练习除外)。
  \mfig{width = 1.00\textwidth}{figs/haskell-quicksort}
\end{problem}

\begin{solution}
  提交方法同上。
\end{solution}
%%%%%%%%%%%%%%%

%%%%%%%%%%%%%%%%%%%%
\beginot

%%%%%%%%%%%%%%%
\begin{ot}[Coq ``FP.v'' Again]
  请结合 \href{https://github.com/hengxin/problem-solving-class-coq/blob/master/2019-1-coq/FP.v}{``FP.v''} 
  介绍函数式程序设计中``高阶函数''的概念与用法。

  \noindent 参考资料:
  \begin{itemize}
    \item Coq \href{https://github.com/hengxin/problem-solving-class-coq/blob/master/2019-1-coq/FP.v}{``FP.v''}
    \item \href{http://learnyouahaskell.com/higher-order-functions}{Higher order functions @ http://learnyouahaskell.com}
    \item 各位助教 (包括 Coq 辅助助教)
  \end{itemize}
\end{ot}

% \begin{solution}
% \end{solution}
%%%%%%%%%%%%%%%
\vspace{0.50cm}
%%%%%%%%%%%%%%%

%%%%%%%%%%%%%%%
\begin{ot}[Coq ``FP.v'' Again and Again]
  请讲解 \href{https://github.com/hengxin/problem-solving-class-coq/blob/master/2019-1-coq/FP.v}{``FP.v''} 
  中最后的``选做练习'' (Church Numerals)。

  \noindent 参考资料:
  \begin{itemize}
    \item Coq \href{https://github.com/hengxin/problem-solving-class-coq/blob/master/2019-1-coq/FP.v}{``FP.v''}
    \item \href{https://en.wikipedia.org/wiki/Church\_encoding}{Church encoding @ wiki}
    \item \href{http://www.cs.unc.edu/~stotts/723/Lambda/church.html}{Natural Numbers as Church Numerals @ cs.unc.edu}
  \end{itemize}
\end{ot}
%%%%%%%%%%%%%%%

%%%%%%%%%%%%%%%%%%%%
% 如果没有需要订正的题目,可以把这部分删掉
\begincorrection

%%%%%%%%%%%%%%%%%%%%

%%%%%%%%%%%%%%%%%%%%
% 如果没有反馈,可以把这部分删掉
\beginfb

% 你可以写
% ~\footnote{优先推荐 \href{problemoverflow.top}{ProblemOverflow}}:
% \begin{itemize}
%   \item 对课程及教师的建议与意见
%   \item 教材中不理解的内容
%   \item 希望深入了解的内容
%   \item $\cdots$
% \end{itemize}
%%%%%%%%%%%%%%%%%%%%
\end{document}